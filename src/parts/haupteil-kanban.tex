\section{Kanban}

\subsection{Einführung}

\frame
{
  \frametitle{Grundsätzliches}
  \setbeamercolor{block title}{fg=white,bg=red!75!black}
  \begin{block}{Definition}
  Kanban is an approach to change management. \\
\smallskip
  It isn’t a software development or project management lifecycle or process.
    \begin{flushright}
    David Anderson / Vater des Software-Kanban (2007)
    \end{flushright}
    \end{block}
\setbeamercolor{block title}{fg=white,bg=blue!75!black}
  \begin{block}{Kanban in a nutshell}
  \begin{itemize}
      \item{Small, cross-functional teams}
      \item{Product split into small, roughly, estimated stories}
      \item{Iterations}
      \item{Continuous improvement}
  \end{itemize}
    \end{block}

}


\frame
{
  \frametitle{Kanban in der Anwendung}
  \begin{itemize}
    \item{Gut geeignet für ungeplante Tätigkeiten: Bugs, Optimierungen, Betriebsprobleme, reaktive Themen, Servicemanagement, \dots}
    \item{Gut geeignet für gut überschaubare Entwicklungsprozesse}
    \item{Begrenze die Menge begonnner Arbeit}
    \item{Verkürze die Zeit zwischen Beginn und Erledigung einer Aufgabe}
    \item{Konzentration und Fokusierung auf die aktuelle Aufgabe}
    \item{Vermeiden von Taskwechsel}
    \item{Visualisiere den Fluss der Arbeit}
    \item{Mache die Regeln für den Prozess explizit\\
      (Publizieren wann wie gehandelt wird)
    }
    \item{Miss und steuere den Fluss}
  \end{itemize}
}


\pgfdeclareimage[width=0.9\textwidth]{kanbanboard2}{../src/images/KanBanBoard-Skizze}
\frame{
   \frametitle{Kanban Board}
       \begin{center}
         \pgfuseimage{kanbanboard2}
         \bigskip
       \end{center}
}

\pgfdeclareimage[width=\textwidth]{kanbanboard}{../src/images/KanBanBoard}
\frame{
   \frametitle{Kanban Board}
       \begin{center}
         \pgfuseimage{kanbanboard}
         \bigskip
       \end{center}
}




