

\section{Scrum}

\subsection{Grundsätzliches}

\pgfdeclareimage[width=1.0\textwidth]{scrum}{../src/images/scrum-bild}
\frame{
  \frametitle{Begriff}
      \begin{center}
         \pgfuseimage{scrum}
       \end{center}
  Scrum (engl. das Gedränge) ist ein Vorgehensmodell mit
  Meetings, Artefakten, Rollen, Werten und Grundüberzeugungen,
  das beim Entwickeln von Produkten im Rahmen agiler
  Softwareentwicklung hilfreich ist.
}

\frame
{
  \frametitle{Wikipedia}
  \begin{itemize}
     \item{OOPSLA 1995, erster Konferenzbeitrag}
     \item{Begriff durch Jeff Sutherland, Ken Schwaber geprägt}
     \item{Umsetzung von Lean Development für das Projektmanagement}
     \item{Kernprinzipien
      \begin{itemize}
           \item{Transparenz: Fortschritt und Hindernisse eine Projektes werden täglich und für sichtbar festgehalten}
           \item{Überprüfung: Regelmäßige Lieferung}
           \item{Anpassung: Anforderungen und Implementierungen entwickeln sich. Neubewertung und Anpassung nach jeder Lieferung}
      \end{itemize}
     }
  \end{itemize}
}


\subsection{Im Detail}

\pgfdeclareimage[width=\textwidth]{scrum}{../src/images/scrum}
\frame{
  \frametitle{Der Scrum Prozess}
       \begin{center}
         \pgfuseimage{scrum}
         \bigskip
       \end{center}
}

\pgfdeclareimage[width=0.68\textwidth]{planungselemente}{../src/images/planungselemente}
\frame{
  \frametitle{Die Planungseinheiten}
       \begin{center}
         \pgfuseimage{planungselemente}
         \bigskip
       \end{center}
}

\pgfdeclareimage[width=\textwidth]{scrum}{../src/images/scrum}
\frame{
  \frametitle{Der Product Owner}
       \begin{center}
         \pgfuseimage{scrum}
         \bigskip
       \end{center}
}

\frame
{
  \frametitle{Der Product Owner}
  \begin{itemize}
    \item{Verwaltet das Produktbacklog}
    \item{Bringt Stories in einen bearbeitbaren Status\\(Erfüllung der ,,Definition of Ready'')}
    \item{Definiert Produkt-Features}
    \item{Priorisiert Features abhängig vom Marktwert}
    \item{Akzeptiert oder weist Arbeitsergebnisse zurück}
    \item{Bestimmt Auslieferungsdatum und Inhalt}
    \item{Ist verantwortlich für das finanzielle Ergebnis des Projekts (ROI)}
    \item{Passt Features und Prioritäten nach Bedarf für jeden Sprint an}
  \end{itemize}
}


\frame{
  \frametitle{Das Team}
       \begin{center}
         \pgfuseimage{scrum}
         \bigskip
       \end{center}
}

\frame
{
  \frametitle{Das Team}
  \begin{itemize}
      \item{Typischerweise 5-9 Personen}
      \item{Funktionsübergreifend / Interdisziplinär \\
        (QS, Programmierer, UI-Designer, System-Engineer, etc.)}
      \item{keine Titel (aber manchmal nicht vermeidbar)}
      \item{Mitglieder sollten Vollzeitmitglieder sein - wenige Ausnahmen \\(z.B. sehr selten benötige Spezialisten)}
      \item{Mitgliedschaft kann sich nur zwischen Sprints verändern}
      \item{Teams organisieren sich selbst}
      \item{Team erfüllt die ,,Definition of Done''}
  \end{itemize}
}

\pgfdeclareimage[width=0.7\textwidth]{deck}{../src/images/deck}
\frame{
   \frametitle{Scrum Planning Poker}
       \begin{center}
          \pgfuseimage{deck}
       \end{center}
       Aufgaben werden durch das Team in einem abstraktem Storypoint-Komplexitätsmaß unabhängig von der Qualifizierung der Teammitglieder geschätzt.
}

\pgfdeclareimage[width=0.3\textwidth]{play}{../src/images/play}
\frame{
   \frametitle{Scrum Planning Poker}
       \begin{center}
          \pgfuseimage{play}
       \end{center}
  \begin{itemize}
      \item{Gemeinsame Einschätzung durch Pokern und Austausch von Argumenten}
      \item{Summe der Komplexitäten = Umfang des Sprints}
      \item{Betrachtung der Leistung des letzten Sprints und Commitment des Teams für bevorstehenden Sprint}
      \item{PO kann mit frühzeitigem Schätzen und der Betrachtung der Storypoints z.B. 1-3 Sprints vorausplanen}
  \end{itemize}
}

\frame{
  \frametitle{Das Daily Standup}
       \begin{center}
         \pgfuseimage{scrum}
         \bigskip
       \end{center}
}

\pgfdeclareimage[width=0.45\textwidth]{standup}{../src/images/standup2}
\frame{
   \frametitle{Daily Standup}
       \begin{center}
         \pgfuseimage{standup}
       \end{center}
   \small
  \begin{itemize}
    \item{Ziel: Die Zusammenarbeit koordinieren und den Arbeitsfluss opimieren\\
      (Reviews, Peering, Hindernisse beseitigen, \dots)}
    \item{Unbedingt vermeiden: Technische Diskussionen, reines Statusreporting, Unpünklichkeit bzw. Nichtanwesenheit}
    \item{Was gehört ins Daily?
     \begin{itemize}
       \item{Gestern habe ich \dots}
       \item{Heute will ich \dots}
       \item{Mich behindert \dots ich komme nicht weiter an \dots weil \dots}
       \item{Ich könnte Unterstützung bei \dots gebrauchen}
     \end{itemize}
    }
  \end{itemize}
}

\pgfdeclareimage[width=\textwidth]{scrumboard}{../src/images/DomaeneIAgileBoard}
\frame{
   \frametitle{Scrum Board I}
       \begin{center}
         \pgfuseimage{scrumboard}
         \bigskip
       \end{center}
}

\pgfdeclareimage[width=0.82\textwidth]{scrumboard2}{../src/images/Scrumboard-physisch}
\frame{
   \frametitle{Scrum Board II}
       \begin{center}
         \pgfuseimage{scrumboard2}
         \bigskip
       \end{center}
}


\pgfdeclareimage[width=0.82\textwidth]{BurndownChart}{../src/images/Burn-down-chart}
\frame{
   \frametitle{Das Burndown Chart}
       \begin{center}
         \pgfuseimage{BurndownChart}
         \bigskip
       \end{center}
}

\frame
{
  \frametitle{Der Scrum Master}
  \begin{itemize}
      \item{Repräsentiert das Management gegenüber dem Projekt}
      \item{Verantwortlich für die Einhaltung von Scrum-Werten und -Techniken}
      \item{Beseitigt Hindernisse}
      \item{Stellt sicher, dass das Team vollständig funktional und produktiv ist}
      \item{Unterstützt die enge Zusammenarbeit zwischen allen Rollen und Funktionen}
      \item{Schützt das Team vor äußeren Störungen}
  \end{itemize}
}

\frame{
   \frametitle{Der Scrum Zyklus}
       \begin{center}
         \pgfuseimage{scrum}
         \bigskip
       \end{center}
}

\frame
{
  \frametitle{Das Sprint Review Meeting}
  \begin{itemize}
      \item{Das Team präsentiert, was es während eines Sprints erreicht hat}
      \item{Der PO gibt einen Ausblick auf den nächsten Sprint}
      \item{Typischerweise in Form einer Demo der neuen Features oder der zugrunde liegenden Architektur}
      \item{Zwei Stunden zur Vorbereitung}
      \item{Keine Folien}
      \item{Das ganze Team nimmt teil}
      \item{Jeder ist eingeladen}
  \end{itemize}
}

\frame
{
  \frametitle{Die Sprint Retrospektive}
  \begin{itemize}
      \item{Nach jedem Sprint}
      \item{bei 3 Wochen Zyklus, ca. 2 Stunden}
      \item{Das ganze Team nimmt teil: Scrummaster, Produktowner, Team}
      \item{Regelgemäßer Rückblick auf Erfolge und Verbesserungsmöglichkeiten}
      \item{Nutzung methodischer Ansätze
          \begin{itemize}
            \item{\href{http://plans-for-retrospectives.com}{Retormat}}
            \item{\href{http://tastycupcakes.org/}{Tastycupcakes}}
          \end{itemize}
      }
      \item{Identifizieren und Auswählen von Sprint-Team-Zielen}
      \item{Nachhalten Sprint-Team-Zielen}
  \end{itemize}
}


\subsection{Auswirkungen}
\frame
{
  \frametitle{Was bringt Scrum dem Fachbereich?}
  \begin{itemize}
    \item{Verlässlichkeit - alle 3 Wochen ein Release}
    \item{keine Change Request Diskussionen mehr}
    \item{Bessere Ergebnisse durch Orientierung am Kunden und am Benutzer}
    \item{Strukturierung in Epics und Stories vereinfacht die Priorisierung}
    \item{Transparenz und kurzfristigere Steuerung} 
    \item{Priorisierung des Backlogs nach Return On Investment und somit Businesswert}
    \item{Kostenkontrolle: ,,Danke. Ich habe ausreichend Funktionalitäten erhalten.''}
    \item{Productowner als Anlaufstelle für Fachlichkeit, die Technik und den Zeitplan}
    \item{Produktentscheidungen zum Zeitpunkt der Relevanz}
    \item{keine Produktentscheidungen ohne Informationen}
  \end{itemize}
}


\frame
{
  \frametitle{Was bringt Scrum dem Entwicklungsteam?}
  \begin{itemize}
    \item{Das Team organisiert sich und seine Arbeit selbst}
    \item{Enge fachliche Zusammenarbeit mit dem Product Owner}
    \item{Konzentration und Fokussierung}
    \item{Besser funktionierende Software durch iterative Vorgehensweise}
    \item{Code wird wartbarer, Bewußtsein dass alles im Fluß ist}
    \item{Bessere Softwarearchitektur - passende Lösungen}
    \item{Anforderungen entwickeln sich sinnvoll und können mitgestaltet werden}
    \item{Kontinuierliche Verbesserung: keine Ownership, Wissenstransfer, Retros, \dots}
    \item{Einfachheit - die Kunst, die Menge nicht getaner Arbeit zu maximieren}
    \item{Definition of Ready: Klare Kriterien für die Implementierungsfähigkeit einer Story}
    \item{Definition of Done: Klare Kriterien für den Abschluß einer Aufgabe}
  \end{itemize}
}




