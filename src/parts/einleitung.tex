\section{Einführung}

\subsection{Der klassische Weg}
\frame
{
  \frametitle{Gründe für Projektunsicherheiten}
  \begin{itemize}
    \item{Kunde weiss am Anfang nicht genau was er braucht}
    \item{Auftragnehmer versteht nicht genau, was der Kunde will}
    \item{Auftragnehmer unterschätzt / überschätzt Aufwände (Planung)}
    \item{Nicht alle Anforderungen und Stakeholder sind bekannt}
    \item{Kunde hat widersprüchliche Anforderungen, z.T. wg. Politik}
    \item{Politik, Management Risiken}
    \item{Änderungen in den Prioritäten, Geschäftsprozessen etc. während des Projektes}
    \item{Projekt in komplexe Projektlandschaft eingebunden}
    \item{Technische Risiken, z.B. Infrastruktur hält nicht was sie verspricht}
  \end{itemize}
}


\pgfdeclareimage[width=1.0\textwidth]{entscheidungsspielraum}{../src/images/entscheidungsspielraum}
\frame{
   \frametitle{Projektfortschritt und Entscheidungsspielraum}
       \begin{center}
         \pgfuseimage{entscheidungsspielraum}
         \bigskip
       \end{center}
}

\frame
{
  \frametitle{Klassische Antworten und Situationen}
  \begin{itemize}
    \item{Anforderungen
      \begin{itemize}
      \item{Ziele: genaues Verständnis der Anforderungen, Anforderungstabilität, (Verfolgbarkeit) }
      \item{Am Anfang des Projekts möglichst vollständig abstimmen \\(Bei Risiken in Stufen vorgehen)}
      \item{Möglichst genaue Spezifikation/Pflichtenheft erstellen}
      \item{Änderungen nur über ein striktes Change Request Verfahren}
      \end{itemize}
    }
    \item{Architektur
      \begin{itemize}
      \item{Ziele: Architektur = stabiles Gerüst der Software}
      \item{In früher Projektphase entwickeln (mindestens Grobarchitektur)}
      \item{Anforderungen späterer Phasen berücksichtigen}
      \end{itemize}
    }
    \item{Dokumente
      \begin{itemize}
      \item{Ziele: Klarheit, Review-Fähigkeit bzw. Abstimmbarkeit, Stabilität und Reduktion von Kommunikation}
      \item{Alles wird dokumentiert (Prinzip der Schriftlichkeit), wichtig sind insbesondere Spezifikationdokumente und Architekturdokumente}
      \item{Für Offshore-Partner, Wartungsteam, das Team}
      \end{itemize}
    }
  \end{itemize}
}


 \frame
 {
   \frametitle{Klassische Antworten und Situationen}
   \begin{itemize}
   \item{Risikomanagement
    \begin{itemize}
     \item{Projekte werden in der Regel in (wenigen) Stufen (3-12 Monate) durchgeführt }
     \item{Aktiver (häufig unsystematischer) Umgang mit Risiken \\
        (Einzelmaßnahmen wie technischer Durchstich, Risikoliste, \dots)}
     \item{Risiken häufig nur pauschal berücksichtigt}
    \end{itemize}
    }
   \item{Pläne
    \begin{itemize}
     \item{Am Anfang des Projektes grobe Planung (auch zur Preisfindung), }
     \item{Feinplanung bei Bedarf}
     \item{Wenn‘s knapp wird: Weglassen/Verschieben unwichtiger Funktionen oder Qualitätseinbußen (z.B. Testaufwände kürzen)}
     \item{Ziele nach Priorität : (1) Einhaltung des Termins, (1) Einhaltung des Budgets, (3) Qualität, (4) Vollständige Funktionalität }
    \end{itemize}
    }
   \item{Verträge
    \begin{itemize}
     \item{Festpreisprojekte zur Kostenkontrolle}
     \item{Preis im Verhältnis zur Funktionalität schwer einschätzbar}
    \end{itemize}
    }
   \end{itemize}
 }
 
 \frame
 {
   \frametitle{Klassische Antworten und Situationen}
   \begin{itemize}
 
   \item{Kunde
    \begin{itemize}
     \item{integriert in Phasen  abhängig von seiner eigenen Verfügbarkeit}
     \item{ist über Anforderungsdefinition, Dokument-Reviews und System- und Abnahmetest einbezogen}
     \item{Entwicklung: selten gemischte Teams bedingt durch Projektphasen}
     \item{Kommunikation zum Teil reglementiert durch Projektleiter („Ein-Ansprechpartner“-Modell) oder über Dokumente}
     \item{Ziele: Langfristige Kundenbindung, Vertrauen, verbindliche Absprachen}
    \end{itemize}
   }
   \item{Prozesse
    \begin{itemize}
     \item{Aktivitäten / Workflows im Detail festgelegt}
     \item{Viele Rollen}
     \item{Viele, genau festgelegte Artefakte (Doku, Code und Co.)}
     \item{Ziele: Planbarkeit, Wiederholbarkeit, Verbesserbarkeit (CMM)}
    \end{itemize}
   }
   \end{itemize}
 }

 \frame
 {
   \frametitle{Auswirkungen}
   \begin{itemize}
      \item{Standard Projekteskalation: Grün \dots Gelb \dots Dunkelrot \dots Gelb} 
      \item{Projektplan und Realität driftet stark auseinander}
      \item{Change Requests machen große Schmerzen (Projektverzug, Kosten)}
      \item{Große Teile der Fachlichkeit ergeben sich während des Projekts}
      \item{Kunde bekommt eigentlich nicht, was er will, sondern vielleicht noch nicht mal das was im Vertrag steht}
      \item{Definierte Prozesse sind oft schwer durchzuhalten}
      \item{Projektziel muß mit massivem Aufwand erkauft werden}
      \item{Projektziel muß mit massiver Scopereduktion erkauft werden}
      \item{Chef-Architekten-Denkmäler}
      \item{weniger Spaß für Projektmitarbeiter}
   \end{itemize}
 }



