\section{Agiles Arbeiten}

\subsection{Allgemein}
\frame
{
  \frametitle{Manifesto for Agile Software Development}
  Was ist uns wichtiger?\\
  \bigskip
\textbf{\textit{Individuals and interactions}}
  \begin{flushright}
  over processes and tools
  \end{flushright}
\textbf{\textit{Working software}}
  \begin{flushright}
  over comprehensive documentation
  \end{flushright}
\textbf{\textit{Customer collaboration}}
  \begin{flushright}
  over contract negotiation
  \end{flushright}
\textbf{\textit{Responding to change}}
\small
  \begin{flushright}
  over following a plan\\
  \end{flushright}
\bigskip
\small
\href{http://agilemanifesto.org}{\beamergotobutton{2001 - http://agilemanifesto.org}}
\normalsize
}

\subsection{Kernaspekte}
\frame
{
  \frametitle{Die agile Vorgehensweise}
  \begin{itemize}
     \item{Das Team übernimmt Verantwortung}
     \item{Ausrichten der Arbeitsweise auf die kontinuierliche Änderung von Anforderungen}
     \item{Probleme iterativ lösen, ,,Minimal viable Product'}
     \item{Bewusst sich um Kommunikation bemühen}
     \item{Ein kontinuierlicher Ergebnisfluß steht im Fokus}
     \item{Produktivität und nicht Effizienz steht im Fokus}
     \item{Konzentration auf möglichst nur eine Aufgabe}
     \item{Stetiger Wissenstransfer, Wissens-Silos vermeiden}
     \item{Arbeiten in Zyklen}
     \item{Transparenz mit allen Beteiligten}
     \item{Konsequente Umsetzung}
     \item{Aktives Arbeiten an der kontinuierlichen Verbesserung}
     \item{Fehlerkultur, Respektvoller und lösungsorientierter Umgang mit Fehlern}
     \item{Ermitteln von KPIs zur Erfolgsmessung}
  \end{itemize}
}

\frame
{
  \frametitle{Auswirkungen auf die Teamkultur}
  \begin{itemize}
     \item{Gemeinsame Verantwortung - kooperativer Führungsstil}
     \item{Das Team organisiert sich weitgehend selbst}
     \item{Führung findet über das Vermitteln von Visionen und Strategie statt}
     \item{Teamleiter unterstützt Team die Aufgaben möglichst eigenverantwortlich zu bearbeiten}
     \item{Entscheidungen werden hauptsächlich im Team gefällt}
     \item{Feedbacks bzw. Verbesserungsprozesse finden regelmäßig und häufig statt (Retro und OneToOne)}
     \item{Kontinuierliches Lernen steht im Vordergrund}
     \item{Peering, gemeinsames Bearbeiten von Problemen}
     \item{Daily Standup Meeting: Was war, was wird, wo gibt es Schwierigkeiten?}
     \item{Persönlicher Kontakt mit allen Beteiligten wird aktiv angestrebt}
     \item{Projektteams bilden sich abteilungsübergreifend}
  \end{itemize}
}



